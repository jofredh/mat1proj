\chapter{Konklusjon}

I dette prosjektet har vi gått igjennom og lært om de matematiske metodene Numerisk tilnærming av den deriverte og Newton-Raphson metoden. Prosjektet har hatt fokus på å utfordre oss innenfor disse områdene og vi har lært ulike konsepter og metoder vi kan bruke til å løse slike problemer på. Vi har blitt bedre kjent med reglene for disse regnemetodene og generelt fått en god forståelse for de forskjellige metodene vi bruker.
\\
\\ 
Etter å ha jobbert med det prosjektet, lærte vi kjente derivasjon regler og bruk av dem for å komme frem til løsningen. I tillegg har vi lært og brukt Limit til å finne en tilnærming til den deriverte av en gitt funksjon når differansen på x-verdiene går mot null. 
\\
\\ 
Gjennom Newton-Raphson metoden lærte vi hvordan vi kunne bruke en funksjon og dens deriverte til å finne en god tilnærming til nullpunktene til funksjonen, så lenge gjennomsnittsveksten ikke er lik null. Andre ting vi oppdaget igjennom oppgaven var at Newton-Raphson metoden ikke alltid fungerer. Det er visse punkter den ikke fungerer i, som for eksempel når $f'(x) = 0$. \\
\\
Alt i alt har dette prosjektet vært en bra lære opplevelse for vår gruppe, og vi har fått stort utbytte av det. Det har vært generelt gode og overkommelige oppgaver som fortsatt ga oss en viss utfordring. 