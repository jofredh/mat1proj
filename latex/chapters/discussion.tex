\chapter{Diskusjon} \\
\textbullet \textbf{Sammenhengen mellom \pmb{$f(x)$} og \pmb{$f'(x)$}. F.eks. hvordan \pmb{$f'(x)$} varierer i forhold til \pmb{$f(x)$}.}\\
\\
$f'(x)$ viser stigningen i $f(x)$ og hvordan den endrer seg, med tanke på x. Vi bruker det hvis vi vil komme frem til hvor mye en funksjon stiger i ett visst punkt. \\
\\
\textbullet \textbf{Hvordan valg av \pmb{$\Delta x$} påvirker nøyaktigheten av tilnærmingene. F.eks. hvor liten kan vi velge \pmb{$\Delta x$} ?\\}
\\
Valget av  $\Delta x$ påvirker nøyaktigheten av tilnærmingene vi gjør. Fordi jo lavere $\Delta x$ vi velger, jo nærmere den faktiske x-verdien er vi. Men når vi velger $\Delta x$ kommer vi til ett punkt hvor det ikke gir mening og velge en lavere verdi. Hvis vi bare vil vise 2 desimaler i svaret gir det ikke mening og velge en $\Delta x > 0.01$.\\
\\
\textbullet \textbf{Hvordan feilen \pmb{$E(x)$} varierer som en funksjon av \pmb{$x$}.\\}
\\
Feilen $E(x)$ vil bli større jo større verdi for x vi putter inn. Ettersom tilnærmingenene og nøyaktigheten vil bli mindre desto mindre verdi for x man velger for $\Delta x$. Hvis funksjonen $f(x)$ ikke er lineær, vil man se at desto lengre vekk man kommer fra det punktet man deriverer i desto større blir feilen.\\ I flere av funksjonene ser vi også en sammenheng mellom $E(x)$ og den annenderiverte $f''(x)$ til funksjonen. De tydeligste eksemplene på dette er Funksjon 1, hvor den annenderiverte er en konstant, og funksjon 2, hvor E, når amplituden blåses opp tilstrekkelig, er en ganske tro gjenskapning av |-sin(x)|, dvs absoluttverdien til den annenderiverte av sin(x).
\\
\textbullet \textbf{Begrensninger ved numeriske tilnærminger av den deriverte.\\}
\\
Begrensningene med den numeriske tilnærmingen av den deriverte er det at vi ikke får en nøyaktig derivert. Altså den vil alltid ha en viss feilmargin. Eneste måten å unngå dette er ved å bruke den eksakte deriverte. \\
\\
\textbullet \textbf{Utforsk den fysiske betydningen av de to løsningene i hvert tilfelle.\\}
\\
Når vi finner nullpunktene til $\theta_2$ får vi 2 verdier, her er den ene verdien for når armen til derricken er bøyd opp, og den andre verdien er når d en er bøyd ned, og den vil aldri bytte mellom disse tilstandene. 
\\
\\
\newpage
\textbullet \textbf{Finn sammenhengen mellom initialverdien brukt i løsningsmetoden, og hvilken av de to løsningene metoden konvergerer mot.\\}
\\
Sammenhengen mellom initialverdien og nullpunktet den konvergerer mot er at hvis du velger initialverdi hvor stigningstallet til funksjonen er negativt konvergerer du mot det de nullpunktet der derricken er bøyd nedover. Velger du initialverdi hvor stigningstallet til funksjonen er positivt konvergerer du mot de nullpunktene hvor derricken er bøyd oppover. Dette gjelder derimot ikke hvis vi velger verdier nærme topp- og bunnpunkter, fordi da er stigningstallet 0.