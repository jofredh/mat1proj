\chapter{Introduksjon}
I dette matematikk prosjektet har vi de siste fire ukene jobbet med å forstå aspekter tilknyttet den deriverte av det vi kaller reelle funksjoner. En reell funksjon er en funksjon hvor både definisjonsmengden og verdimengden til funksjonen er reelle tall. Vi har også jobbet med å kunne forstå hvordan man skal kunne bruke Newton - Raphson metoden for å kunne finne tilnærminger til nullpunkter av ikke - lineære likninger. Vi har utforsket forskjellige måter å løse disse oppgavene på, og har endt opp med å bruke flere forskjellige midler til å løse oppgavene i dette prosjektet. Vi har løst noen oppgaver for hånd, mens andre oppgaver har vi brukt programmer som GeoGebra og også laget egne programmer i kodespråket Python til å finne løsninger til oppgavene. Vi har lært mye i løpet av dette prosjektet og vi skal i denne rapporten presentere teorien bak matematikken i prosjektet og presentere våre metoder og resultater. 
\\



